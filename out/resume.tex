% !TeX program = xelatex
\documentclass[a4paper]{article}
    \usepackage{fullpage}
    \usepackage{amsmath}
    \usepackage{amssymb}
    \usepackage{textcomp}
    \usepackage{fontspec}
    \usepackage[utf8]{inputenc}
    \usepackage[T1]{fontenc}
    \usepackage{hyperref}
    \usepackage{fancyhdr}
    \textheight=10in
    \raggedright
    \setmainfont{Arimo}
    \setmonofont{Arimo}
    \hypersetup{colorlinks=true,linkcolor=blue,urlcolor=[rgb]{0,0.2,0.6}}
    \thispagestyle{fancy}
    \fancyhf{}
    \renewcommand{\headrulewidth}{0pt}



    %\renewcommand{\encodingdefault}{cg}
%\renewcommand{\rmdefault}{lgrcmr}

\def\bull{\vrule height 0.8ex width .7ex depth -.1ex }

% Initial template generated by https://latexresu.me/
% DEFINITIONS FOR RESUME %%%%%%%%%%%%%%%%%%%%%%%

\newcommand{\area} [2] {
    \vspace*{-9pt}
    \begin{verse}
        \textbf{#1}   #2
    \end{verse}
}

\newcommand{\lineunder} {
    \vspace*{-8pt} \\
    \hspace*{-18pt} \hrulefill \\
}

\newcommand{\header} [1] {
    {\hspace*{-18pt}\vspace*{6pt} \textsc{#1}}
    \vspace*{-6pt} \lineunder
}

\newcommand{\employer} [3] {
    { \textbf{#1} (#2)\\ \underline{\textbf{\emph{#3}}}\\  }
}

\newcommand{\contact} [3] {
    \vspace*{-10pt}
    \begin{center}
        {\Huge \scshape {#1}}\\
        #2 \\ #3
    \end{center}
    \vspace*{-8pt}
}

\newenvironment{achievements}{
    \begin{list}
        {$\bullet$}{\topsep 0pt \itemsep -2pt}}{\vspace*{4pt}
    \end{list}
}

\newcommand{\schoolwithcourses} [4] {
    \textbf{#1} #2 $\bullet$ #3\\
    #4 \\
    \vspace*{5pt}
}

\newcommand{\school} [4] {
    \textbf{#1} #2 $\bullet$ #3\\
    #4 \\
}
% END RESUME DEFINITIONS %%%%%%%%%%%%%%%%%%%%%%%

\newcommand{\github} [1] {\href{https://github.com/#1}{#1}}

    \begin{document}
\vspace*{-40pt}

%==== Profile ====%
\vspace*{-10pt}
\begin{center}
	{\Huge \scshape {Thai Pangsakulyanont}}\\ \vspace{4pt}
	\href{https://github.com/dtinth}{@dtinth} $\cdot$ Bangkok, TH $\cdot$ dtinth@spacet.me $\cdot$ \url{https://dt.in.th/}\\
\end{center}

%==== Experience ====%
\header{Work Experience}
\vspace{1mm}




\textbf{Taskworld} \hfill Bangkok, TH\\
\textit{Frontend Architect} \hfill Jun 2015 – Present\\
\vspace{-1mm}
\begin{itemize} \itemsep 1pt
    
	\item Set up and evolved our end-to-end testing system, from Nightwatch (JavaScript) to RSpec and Capybara (Ruby), to ultimately creating \github{taskworld/prescript}, our own test runner written in TypeScript. It allows us to run tests in parallel, generate detailed test reports, and provides tools to help us debug end-to-end tests more easily.
    
	\item Implemented a browser automation abstraction layer in our test suite. 2 years later, this abstraction would allow us to migrate our tests from Selenium WebDriver to Playwright with very little change to the test scripts.
    
	\item Migrated the frontend application build pipeline from Browserify to webpack, giving us access to modern performance optimizations techniques such as tree-shaking and code splitting, helping us to make our application load faster.
    
	\item Replaced a jQuery-based kanban board component with one written from scratch on top of React to support touchscreens and give us more control over the UI and UX.
    
	\item Improved application observability in the frontend by integrating with Real User Monitoring (RUM) services to help engineers troubleshoot application errors more easily.
    
	\item Improved application observability in the backend by implementing diagnostics logging, structured logging, and distributed tracing. These added logs would become instrumental in providing us vital clues to resolve various issues that happen only on production.
    
	\item Evolved frontend application’s continuous delivery pipeline based on GitHub Flow, giving engineers enough confidence to deploy frontend code to production environments on demand (i.e. multiple times a day).
    
	\item Set up systems to speed up the continuous delivery pipeline by caching build outputs, generating automatic changelogs, and allowing multiple small PRs to be batched up, tested and merged together.
    
	\item Created custom ESLint plugins to ensure consistent coding conventions are used across the codebase, saving us time and effort in reviewing PRs, and encouraged other engineers to do the same. \github{taskworld/eslint-plugin-local}
    
	\item Worked on building product features and fixing bugs while remaining focused on long-term maintainability and developer experience on the codebase. Also built some quick hacks (that are easy to delete!) in response to business needs, such as prototyping an integration for Google Meet to improve collaboration during COVID-19.
    
	\item Gave a few talks based on my working experience at Taskworld: \href{https://dt.in.th/smells-in-react-apps.html}{“Smells in React Apps”}, \href{https://dt.in.th/race-conditions-in-js-apps.html}{“Race Conditions in JavaScript Apps”}, and \href{https://dt.in.th/embracing-gradual-typing.html}{“Embracing gradual typing”}.
    
\end{itemize}

\textbf{OOZOU} \hfill Bangkok, TH\\
\textit{Intern} \hfill Aug 2014 – Dec 2014\\
\vspace{-1mm}
\begin{itemize} \itemsep 1pt
    
	\item Worked on building products, fixing bugs, converting designs to code, delivering value. You know, the usual software developer stuff.
    
	\item Set up Jasmine test framework for a Ruby on Rails project that lacks frontend testing.
    
	\item Improved testing experience in JavaScript by building the \github{oozou/stateful-context} library.
    
	\item Investigated and fixed performance bottlenecks in Ruby using flamegraphs, creating the \github{oozou/ruby-prof-flamegraph} gem in the process.
    
	\item Wrote several articles on OOZOU blog: \url{https://oozou.com/blog/authors/thai}
    
\end{itemize}



%==== Education ====%
\header{Education}
\vspace{1mm}

\textbf{Kasetsart University}\\
Software and Knowledge Engineering \hfill Jun 2011 – May 2015\\
\vspace{2mm}


\cfoot{\textit{For more software projects, talks, and songs, please check out my website at \url{https://dt.in.th/}.}}

% New page
\newpage

\begin{center}
    \textit{The following pages are about what I do outside of work. \\ They are rather long but in case you’re interested…}
\end{center}
\vspace{3mm}






\header{Community projects}
\vspace{1mm}

\textbf{Zone-IT (2006 – 2010)}\\
\textit{A web forum about all things IT, with over 100 categories, 100,000 users and 500,000 posts. I was part of the administrator team, and I mainly helped with technical aspects of the project.}

\begin{itemize} \itemsep 1pt
    
	\item Heavily modified the Simple Machines Forum (PHP+MySQL) to add features which set it apart from other forums: implementing a chat system, custom BBCode tags, radio station system, portal pages, and other features, using XMLHttpRequest and MooTools heavily.
    
	\item Designed 3 major iterations of the forum look and feel. Added various microinteractions and animations to the forum. (CSS animations and transitions didn’t exist back then; we used MooTools.Fx.)
    
	\item Optimized the forum’s code, added layers of caching, and tuned the system and database performance to be able to serve over 700 simultaneous users.
    
	\item Performed archive tasks to convert old posts into static HTML pages, in order to reduce the load on the server.
    
\end{itemize}

\vspace{2mm}

\textbf{React Bangkok (2016 – 2019)}\\
\textit{A series of meetups and conferences focused on React.}

\begin{itemize} \itemsep 1pt
    
	\item Gave a few talks: \href{https://dt.in.th/higher-order-components-and-recompose-talk.html}{“HOCs and Recompose”} and \href{https://dt.in.th/reactbkk2-live-coding.html}{“Live coding and mob programming session”}
    
	\item Helped in organizing the events, mainly tasks that require coding: working on event websites, GitHub issue bot, custom ticket check-in systems, name tag generation system, and live Twitter board.
    
\end{itemize}

\vspace{2mm}

\textbf{JavaScript Bangkok (2020)}\\
\textit{An international conference about JavaScript.}

\begin{itemize} \itemsep 1pt
    
	\item Helped in architecting and remotely collaborating on the conference website using Figma and CodeSandbox. The process of building the website is livestreamed and I posted a video to document our \href{https://youtu.be/watch?v=uQH2R-BE1lw}{synchronous remote collaboration} on YouTube.
    
	\item Implemented lunchtime meal reservation system using Firebase, ensuring that the system can handle 800 people reserving their meal at the same time, keeping all data consistent.
    
	\item Live-translated Thai talks into English during the event.
    
\end{itemize}

\vspace{2mm}

\textbf{The Stupid Hackathon Thailand (2017 – 2021)}\\
\textit{A series of hackathons to encourage people to just have fun building stuff. It’s one of the funniest hackathons in Thailand. Helped in organizing the 1st to 4th hackathons. For the 5th hackathon, I was a consultant for the organizing team.}

\begin{itemize} \itemsep 1pt
    
	\item Gave talks about the art of rapid prototyping and provided technical mentorship to the participants (1st).
    
	\item Devised schemes to award hackathon prizes to participants in a non-competitive way (2nd-4th).
    
	\item Devised a series of trivia challenges that one has to solve in order to get a ticket to the hackathon due to high interest but limited venue capacity (2nd–3rd).
    
	\item Created an art direction and produced graphic assets for the hackathons (3rd–4th).
    
	\item Went virtual due to COVID-19. Set up virtual spaces and planned out activities during the hackathon (4th).
    
	\item Helped in organizing other events by the same team, such as Hacktoberfest Bangkok meetup 2020 and BKK.JS \#14.
    
	\item Made a few projects myself: \href{https://dt.in.th/bangkokipsum.html}{Bangkok Ipsum}, \href{https://dt.in.th/midi-light-switch.html}{MIDI Light Switch} and \href{https://www.youtube.com/watch?v=2JE3DETHTQo}{Misheard}.
    
\end{itemize}

\vspace{2mm}

\textbf{Code in the Dark Thailand (2018 – 2019)}\\
\textit{An challenging frontend contest event by Tictail. For more info see \href{http://codeinthedark.com/}{codeinthedark.com}. Being an open-sourced event, I helped bringing it to Thailand. We ran 3 events in Bangkok and Chiang Mai.}

\begin{itemize} \itemsep 1pt
    
	\item Came up with web design implementation challenges for the participants (1st, 2nd).
    
	\item Devised simple challenges that one has to solve in order to get a ticket to the hackathon due to high interest but limited venue capacity (1st).
    
	\item Came up with a frontend quiz for the audience. The quiz result is used to select who would become a contestant. This replaced the first-come-first-served sign-up system which I believe prefers confidence over skills (3rd).
    
\end{itemize}

\vspace{2mm}

\textbf{ELECT LIVE (2019)}\\
\textit{An open-source website by the ELECT.in.th initiative that shows the live results of 2019 Thailand Election. I volunteered as a frontend architect.}

\begin{itemize} \itemsep 1pt
    
	\item Helped in building the technical architecture of the website using React, Gatsby, Emotion and MobX. The architecture provided tools for developers to inspect the application state and raw data from within the app, allowing for easier debugging.
    
	\item Used the \href{https://www.yegor256.com/2010/03/04/pdd.html}{Puzzle-Driven Development} methodology to break down programming tasks into smaller and easier tasks which can be independently worked on. This eventually allowed over 20 people to contribute to the website, closing over 100 issues in just 8 days.
    
	\item Documented our learnings in \href{https://wonderful.software/elect-live/}{ELECT LIVE development notes} (Thai) and on the \github{electinth/election-live} repository’s README file (English).
    
\end{itemize}

\vspace{2mm}

\textbf{wonderful.software Webring (2021)}\\
\textit{A webring for Thai developers, designers, and artists. Each site in the ring links to the main webring site. The webring then takes the visitor to the next site in the ring, allowing traffic to flow between members’ sites. See \href{https://webring.wonderful.software/}{webring.wonderful.software}.}

\begin{itemize} \itemsep 1pt
    
	\item Built the webring using HTML, CSS and Vue.js using no build tooling. This lets people add their site to the project without having to install anything.
    
	\item Created an end-to-end test suite for the website using Cypress. As a website that I rarely maintain, I don’t want to spend much time manually testing it when making changes.
    
	\item Created a semi-automated PR review workflow using GitHub Actions. It automatically visits the website and checks for a valid backlink, saving me time when reviewing PRs.
    
	\item Set up automation to continually update the website screenshots in the webring using Puppeteer.
    
	\item Set up system to collect and publish webring usage stats using Amplitude, Google Cloud Functions, and GitHub Actions.
    
\end{itemize}

\vspace{2mm}

\textbf{Clubhouse Music Jamming Room (2021)}\\
\textit{A community of people who like to jam together online using the open-source \href{https://jamulus.io/}{Jamulus} software. We used Clubhouse to broadcast our jam sessions live and find new members.}

\begin{itemize} \itemsep 1pt
    
	\item Set up and maintained the Jamulus servers for the group.
    
	\item Managed a fleet of VPS servers using Ansible as we make our servers public and adding more servers to our fleet. Created tools to centrally manage the servers using PHP (on the C\&C server) and Python (on each Jamulus server).
    
	\item Set up server monitoring using Azure Monitor.
    
	\item Evaluated cloud service provider performance by signing up for a bunch of cloud services and putting them through a series of benchmarks (audio encoding test, packet delay variation (jitter) test). The results would then be analyzed to find out which cloud service provider provides the best service with reasonable price.
    
	\item Contributed a patch to introduce \href{https://github.com/jamulussoftware/jamulus/pull/1975}{JSON-RPC} to the Jamulus project as a way to programmatically control the Jamulus servers and clients.
    
\end{itemize}

\vspace{2mm}



\header{Personal projects}
\vspace{1mm}

\textbf{Personal digital infrastructure project (2018 – Present)}\\
\textit{A personal project to build tools that lets me be more lazy.}

\begin{itemize} \itemsep 1pt
    
	\item Created a serverless rapid prototyping platform on Google Cloud Run: \github{dtinth/evalaas}. It lets me prototype and make changes to the chat bot and see the results immediately, without having to go through a full Google Cloud Run service deployment cycle (which takes minutes).
    
	\item Set up a multi-channel chat bot (LINE, Slack, CLI) that automates my home, tracks my expenses, lets me take notes, and make quick calculations: \github{dtinth/automatron}.
    
	\item Built an Android app to send notifications from my phone to automatron, so that financial transactions can be automatically tracked: \github{dtinth/dtinth.tools-android}.
    
	\item Set up an automated script to periodically take screenshots of my web projects: \github{dtinth/timelapse}.
    
	\item Set up a system that lets me quickly take notes in Markdown from inside VS Code, and share them as a webpage: \github{dtinth/notes.dt.in.th}.
    
	\item Set up an Antora-based multi-repository documentation site that lets me document my projects without having to create a separate documentation site for each one: \github{dtinth/docs}.
    
\end{itemize}

\vspace{2mm}





\header{Technologies}

\textit{This is an alphabetical list of technologies, tools, and software that I have some experience with, sourced from my past projects:}

\vspace{2mm}

Airtable API, Allure Framework, Amazon Lightsail, Amazon S3, Amplitude, AngularJS, Antora, API Documenter, API Extractor, AsciiDoc, Auth0, AWS Cloud9, Axios, Azure Blob Storage, Azure Monitor, Babel, Browserify, Canvas API, Chai, Chrome Extension, CircleCI, Cloudflare CDN, CodeSandbox, CoffeeScript, CSS, Cucumber, curl, Cypress, Datadog, Debian, DeviceOrientation API, DigitalOcean App Platform, Discord.js, Docker, Docker Compose, Elasticsearch, Electron, Emotion, ESLint, execa, Express, Facebook Graph API, FFmpeg, Figma, Firebase Authentication, Firebase Cloud Firestore, Firebase Hosting, Firebase Realtime Database, Fish Shell, getDisplayMedia API, getUserMedia API, Git, GitHub Actions, GitHub API, GitHub CLI, GitHub Codespaces, GitHub Packages, GitHub Pages, Glitch.com, Google APIs, Google Artifact Registry, Google Cloud Run, Google Cloud Speech API, Google Cloud Storage, Google Container Registry, Google Drive API, Google Kubernetes Engine, Google Sheets API, Google Sign-In, Gulp, Handlebars, Heroku, Hot Module Replacement, HTML Custom Elements, Huawei Cloud ECS, Icecast2, Immutable.js, JavaScript, Jekyll, Jest, jQuery, JSON-RPC, JWT, JXA, Karma, Keynote, Less CSS, LINE Bot, Linode, Linux, Lodash, Markdown, MobX, Mocha, MongoDB, MongoDB Atlas, n-gram Language Modeling, ncc, Netlify, Netlify Functions, Next.js, Node.js, Notion, npm, Nuxt.js, OAuth2, Octokit, PHP, Pino, pnpm, postMessage API, PouchDB, Prescript, Prettier, Puppeteer, PWA, Python, React, Redux, Reselect, RobotJS, Ruby, RxJS, SCSS, Selenium WebDriver, Service Workers, Shell Script, Slack API, Socket.io, Styled Components, Stylus, Svelte, SVG, Tailwind CSS, Test Automation, Three.js, tmux, TweetNaCl, Twind, Twitter API, TypeScript, Ubuntu, UMD, Userscript, Vercel, Vim, Visual Studio Code, Vite, VS Code Extension API, Vue.js, VuePress, Web Audio API, Web Gamepad API, Web MIDI API, Web Share Target API, Web Speech API, webpack, WebRTC (simple-peer), WebSocket, YAML, Yargs, Yarn, Yjs.

\vspace{2mm}


\textit{The following technologies are also mentioned in my GitHub repositories, but I haven’t used them enough (or haven’t used them for too long — many technologies on this list are already dead):}

\vspace{2mm}

Adobe AIR, Android SDK, Apollo Server, AppleScript, Applitools, Atom, AudioWorklet, AutoIt, AutoPy, Batch Script, Blurhash, Bower, C, C++, Caddy Server, Chrome DevTools Protocol, ClojureScript, Cloudflare Workers, Cocoa, CodeMirror, Docsify, Docusaurus, Ext JS, Fastify, Firefox Extension API, Flutter, FMOD, Gauge, GLSL, Google App Engine, Google Cloud IoT Core, GraphQL, Hammerspoon, Haxe, Heft, IPFS, Jasmine, Java, Jupyter Notebook, Kotlin, Kubernetes, Lerna, LevelDB, libsndfile, Lua, Metalsmith, MooTools, MQTT, Music Theory, OAuth 1.0a, Objective-C, OpenAI API, opentype.js, Opus, Pandoc, Philips Hue API, PostCSS, Preact, PulseAudio, Qt, Redux Toolkit, RefluxJS, remark, RequireJS, RtMidi, Rust, Serde, SonarCloud, SpiderMonkey, Stencil.js, StratifiedJS, Surge.sh, Swift, Tokio, Tornado Web Server, Travis CI, unist, Visual Basic .NET, Visual Basic 6, Vuetify, Web Share API, WebAssembly, WebGL, WebRTC, Wintersmith, x86 Assembly, XMPP, Xterm.js, XULRunner, Xvfb, YouTube Player API, YQL, ZeroMQ.

\vspace{2mm}





\ 
\end{document}